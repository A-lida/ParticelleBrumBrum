\documentclass{article}
\usepackage{graphicx} 
\usepackage{amsmath}
\usepackage{listings}
\usepackage{hyperref}
\usepackage[dvipsnames]{xcolor}
\usepackage[top=7cm, bottom=7cm, left=2.5cm, right=2.5cm]{geometry}
\geometry{a4paper, margin=1in, top=4cm, bottom=4cm, left=2.5cm, right=2.5cm}
\title{\textbf{Relazione di Laboratorio: simulazione di un esperimento di fisica delle alte energie}}
\author{ Chiara Baldelli, Elisa Barilari, Alida Castagnoli, Ilaria Core}
\date{? 2023}
\begin{document}
\maketitle
\tableofcontents
\newpage

\section{Struttura del programma} Il programma realizzato vuole simulare un esperimento di fisica delle alte energie, analizzando le collisioni di particelle elementari di tipo Kaone (+/-/*), Pione (+/-) e Protone (+/-). Si compone di 4 file source (.cpp) inseriti nella cartella \verb|main/src|, 3 file header (.hpp) inseriti nella cartella \verb|main/include| e di un tool di compilazione, CMake, comprendente diversi file nella cartella \verb|project-compilation| e diversi file CMakeLists.txt. Inoltre, per testare i metodi implementati, abbiamo realizzato con il framework Doctest un file di testing nella cartella \verb|test/unit/sample|. Fanno parte del programma anche un file README.md, contenete una breve descrizione del progetto; e un file Relazione.tex realizzato con Latex/texmaker contenente la relazione. Per la stesura del codice ci siamo avvalsi del tool git e di una repository remota su github, disponibile all'indirizzo https://github.com/Chiarass/ParticelleBrumBrum